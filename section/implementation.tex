\section{Face Recognition with OpenCV2 C++ API}

The C++ API of OpenCV2 closely resembles the \ifx\python\undefined \href{http://www.gnu.org/software/octave/}{GNU Octave}/\href{http://www.mathworks.com}{MATLAB} \else \href{http://www.python.org}{Python} \fi code we've written in section \ref{ssection:example_eigenfaces} and \ref{ssection:example_fisherfaces}. If you have prototyped it, you won't have a great problem translating it to OpenCV2. I personally think you don't need a book to learn about the OpenCV2 C++ API. There's a lot of documentation coming with OpenCV, just have a look into the \lstinline|doc| folder of your OpenCV installation. The easiest way to get started is the \textit{OpenCV Cheat Sheet (C++)} (\lstinline|opencv_cheatsheet.pdf|), because it shows you how to use all the functions with examples. The \textit{The OpenCV Reference Manual} (\lstinline|opencv2refman.pdf|) is the definite guide to the API (500+ pages). Of course, you'll need a book or other literature for understanding computer vision algorithms - I can't give an introduction to this here.

\subsection{Downloading and Building the Source Code}

Only a high-level description of the implementation is given, because pasting the complete code doesn't make any sense. The source code is available at \url{http://www.github.com/bytefish/opencv} and none of the projects have additional dependencies.

\begin{itemize}
	\item \textbf{Eigenfaces}: \url{https://github.com/bytefish/opencv/tree/master/eigenfaces}
	\item \textbf{Fisherfaces}: \url{https://github.com/bytefish/opencv/tree/master/lda}
\end{itemize}

If you want to clone both projects with \href{http://git-scm.com/}{git} then issue:

\begin{lstlisting}
git clone git@github.com:bytefish/opencv.git
\end{lstlisting}

However, if you don't have \href{http://git-scm.com/}{git} on your system you can download both projects as zip or tarball:

\begin{itemize}
	\item \textbf{zip} \url{https://github.com/bytefish/opencv/zipball/master}
	\item \textbf{tar} \url{https://github.com/bytefish/opencv/tarball/master}
\end{itemize}

Building the demo executables is then as simple as (assuming you are in a projects folder):

\begin{lstlisting}
mkdir build
cd build
cmake ..
make
./eigenfaces /path/to/csv.ext
\end{lstlisting}

Or if you are on Windows with MinGW you would do:

\begin{lstlisting}
mkdir build
cd build
cmake -G "MinGW Makefiles" ..
mingw32-make
eigenfaces.exe /path/to/csv.ext
\end{lstlisting}

Please make sure to read the \textit{README.markdown} coming with each project!

\subsection{Reading in the face images}

\lstset{language=,}

OpenCV can read data from various video sources and image types, you can start your research with the documentation on \href{http://opencv.willowgarage.com/documentation/cpp/reading_and_writing_images_and_video.html}{Reading and Writing Images and Video}. I needed to read images from different folders for a project and I don't know a simpler approach than reading from a CSV file (if you know a simpler approach, please ping me):

\begin{lstlisting}[language=c++]
void read_csv(const string& filename, vector<Mat>& images, vector<int>& labels) {
  std::ifstream file(filename.c_str(), ifstream::in);
  if(!file)
    throw std::exception();
  std::string line, path, classlabel;
  // for each line
  while (std::getline(file, line)) {
    // get current line
    std::stringstream liness(line);
    // split line
    std::getline(liness, path, ';');
    std::getline(liness, classlabel);
    // push pack the data
    images.push_back(imread(path,0));
    labels.push_back(atoi(classlabel.c_str()));
  }
}
\end{lstlisting}

Some people had questions about the usage of the executables, mainly concerned with reading the images and the corresponding labels from a CSV file. Basically all the CSV file needs to contain are lines composed of a \textit{filename} followed by a \textit{;} followed by the \textit{label} (as integer number), making up a line like this: \lstinline|/path/to/image;0|. So if the AT\&T Facedatabase is extracted to \lstinline|/home/philipp/facerec/data/at| the CSV file has to look like this:

\begin{lstlisting}
/home/philipp/facerec/data/at/s1/1.pgm;0
/home/philipp/facerec/data/at/s1/2.pgm;0
[...]
/home/philipp/facerec/data/at/s2/1.pgm;1
/home/philipp/facerec/data/at/s2/2.pgm;1
[...]
/home/philipp/facerec/data/at/s40/1.pgm;39
/home/philipp/facerec/data/at/s40/2.pgm;39
\end{lstlisting}

Think of the label as the subject (the person) you want to recognize. You don't need to take care about the order of the labels, just make sure the same subjects (persons) belong to the same (unique) label.\footnote{The CSV file for the AT\&T Database comes with this document.} I'll now show the definition of the classes and a source code listing, that shows how to use the classes. I think the code answers most of the questions already.

\subsection{Eigenfaces}

\subsubsection{Definiton}
\begin{lstlisting}[language=c++]
class Eigenfaces {
public:
  //! Initialize an empty Eigenfaces model with num_components = 0 
  //   and dataAsRow = true.
  Eigenfaces();
  //! Initialize a Eigenfaces model for num_components and dataAsRow.
  Eigenfaces(int num_components, bool dataAsRow = true);
  //! compute the eigenfaces for data (given in src) and labels, keep 
  //   num_components principal components. Pass dataAsRow = true, if 
  //   the observations are given by row, false if given by column.
  Eigenfaces(const vector<Mat>& src, 
      const vector<int>& labels, 
      int num_components = 0, 
      bool dataAsRow = true);
  //! compute the eigenfaces for data (given in src) and labels, keep 
  //   num_components principal components. Pass dataAsRow = true, if 
  //   the observations are given by row, false if given by column.
  Eigenfaces(const Mat& src, 
      const vector<int>& labels, 
      int num_components = 0, 
      bool dataAsRow = true);
  //! compute the eigenfaces for data (given in src) and labels      
  void compute(const vector<Mat>& src, const vector<int>& labels);
  //! compute the eigenfaces for data (given in src) and labels
  void compute(const Mat& src, const vector<int>& labels);
  //! get a prediction for a given a sample
  int predict(const Mat& src);
  //! project a sample
  Mat project(const Mat& src);
  //! reconstruct a sample
  Mat reconstruct(const Mat& src);
  //! getter
  Mat eigenvectors() const;
  Mat eigenvalues() const;
  Mat mean() const;
};
\end{lstlisting}

\subsubsection{Example}
\begin{lstlisting}[language=c++]
// ...
// include the eigenfaces
#include "eigenfaces.hpp"
// ...
int main(int argc, char *argv[]) {
  // the samples and corresponding labels (classes/subjects/...)
  vector<Mat> images;
  vector<int> labels;
  // read in images and labels
  string fn_csv = string("/path/to/your/csv.ext");
  // read in the images
  try {
    read_csv(fn_csv, images, labels);
  } catch(exception& e) {
    cerr << "Error opening file \"" << fn_csv << "\"." << endl;
    exit(1);
  }
  // get width and height of the samples
  int width = images[0].cols;
  int height = images[0].rows;
  // get a test sample
  Mat testSample = images[images.size()-1];
  int testLabel = labels[labels.size()-1];
  // ... and delete it from training samples
  images.pop_back();
  labels.pop_back();
  // num_components eigenfaces 
  int num_components = 80;
  // compute the eigenfaces
  Eigenfaces eigenfaces(images, labels, num_components);
  // get a prediction (recognize a face)
  int predicted = eigenfaces.predict(testSample);
  cout << "actual=" << testLabel << " / predicted=" << predicted << endl;
  // see the reconstruction with num_components
  Mat p = eigenfaces.project(images[0].reshape(1,1));
  Mat r = eigenfaces.reconstruct(p);
  // see the reconstruction with num_components eigenfaces
  imshow("original", images[0]);
  imshow("reconstruction", toGrayscale(r.reshape(1, height)));
  // get the eigenvectors
  Mat W = eigenfaces.eigenvectors();
  // show first 10 eigenfaces
  for(int i = 0; i < 10; i++) {
    Mat ev = W.col(i).clone();
    imshow(num2str(i), toGrayscale(ev.reshape(1, height)));
  }
  // ...
}
\end{lstlisting}

\subsection{Fisherfaces}

\subsubsection{Definition}
\begin{lstlisting}[language=c++]
class Fisherfaces {

public:
  //! Initialize an empty Eigenfaces model with num_components = 0
  //   and dataAsRow = true.
  Fisherfaces();
  //! Initialize a Fisherfaces model for num_components and dataAsRow.
  Fisherfaces(int num_components, 
      bool dataAsRow = true);
  //! compute the fisherfaces for data (given in src) and labels, keep 
  //   num_components discriminants. Pass dataAsRow = true, if the 
  //   observations are given by row, false if given by column.
  Fisherfaces(const vector<Mat>& src, 
      const vector<int>& labels,
      int num_components = 0,
      bool dataAsRow = true);
  //! compute the fisherfaces for data (given in src) and labels, keep 
  //   num_components discriminants. Pass dataAsRow = true, if the 
  //   observations are given by row, false if given by column.
  Fisherfaces(const Mat& src,
      const vector<int>& labels,
      int num_components = 0,
      bool dataAsRow = true);
  //! compute the fisherfaces for data (given in src) and labels 
  void compute(const Mat& src, const vector<int>& labels);
  //! compute the fisherfaces for data (given in src) and labels 
  void compute(const vector<Mat>& src, const vector<int>& labels);
  //! get a prediction for a given a sample
  int predict(const Mat& src);
  //! project a sample
  Mat project(const Mat& src);
  //! reconstruct a sample
  Mat reconstruct(const Mat& src);
  // getter
  Mat eigenvectors() const;
  Mat eigenvalues() const;
  Mat mean() const;
};
\end{lstlisting}

\subsubsection{Example}
\begin{lstlisting}[language=c++]
// ...
// include the fisherfaces header
#include "fisherfaces.hpp"
// ...
int main(int argc, char *argv[]) {
  // the samples and corresponding labels (classes/subjects/...)
  vector<Mat> images;
  vector<int> labels;
  // read in images and labels
  string fn_csv = string("/path/to/your/csv.ext");
  // read in the images
  try {
    read_csv(fn_csv, images, labels);
  } catch(exception& e) {
    cerr << "Error opening file \"" << fn_csv << "\"." << endl;
    exit(1);
  }
  // get width and height
  int width = images[0].cols;
  int height = images[0].rows;
  // get test instances
  Mat testSample = images[images.size()-1];
  int testLabel = labels[labels.size()-1];
  // ... and delete last element
  images.pop_back();
  labels.pop_back();
  // build the Fisherfaces model
  subspace::Fisherfaces model(images, labels);
  // test model
  int predicted = model.predict(testSample);
  cout << "predicted class = " << predicted << endl;
  cout << "actual class = " << testLabel << endl;
  // get the eigenvectors
  Mat W = model.eigenvectors();
  // show first 10 fisherfaces
  for(int i = 0; i < 10; i++) {
    Mat ev = W.col(i).clone();
    imshow(num2str(i), toGrayscale(ev.reshape(1, height)));
  }
}
\end{lstlisting}

\subsection{Linear Discriminant Analysis}

The Fisherfaces method includes a Linear Discriminant Analysis, so you get this class for free.

\subsubsection{Definition}
\begin{lstlisting}[language=c++]
class LinearDiscriminantAnalysis {

public:
  //! Initialize an empty LDA model with num_components = 0
  //   and dataAsRow = true.
  LinearDiscriminantAnalysis();
  //! Initialize a LDA model for num_components and dataAsRow.
  LinearDiscriminantAnalysis(int num_components, bool dataAsRow = true);
  //! compute the LDA for data (given in src) and labels, keep 
  //   num_components discriminants. Pass dataAsRow = true, if the 
  //   observations are given by row, false if given by column.
  LinearDiscriminantAnalysis(const Mat& src,
      const vector<int>& labels,
      int num_components = 0,
      bool dataAsRow = true);
  //! compute the LDA for data (given in src) and labels, keep 
  //   num_components discriminants. Pass dataAsRow = true, if the 
  //   observations are given by row, false if given by column.
  LinearDiscriminantAnalysis(const vector<Mat>& src,
      const vector<int>& labels,
      int num_components = 0,
      bool dataAsRow = true);
  //! compute the LDA for data (given in src) and labels 
  void compute(const Mat& src, const vector<int>& labels);
  //! compute the LDA for data (given in src) and labels 
  void compute(const vector<Mat>& src, const vector<int>& labels);
  //! project a sample
  Mat project(const Mat& src);
  //! reconstruct a sample
  Mat reconstruct(const Mat& src);
  //! getter
  Mat eigenvectors() const;
  Mat eigenvalues() const;
};
\end{lstlisting}

\subsubsection{Example}
This example is the OpenCV C++ implementation of the tutorial at \url{http://bytefish.de/wiki/pca_lda_with_gnu_octave}. The values found by GNU Octave are reported in the comments. If you want to work through the example yourself, the GNU Octave/MATLAB code is \href{http://bytefish.de/wiki/pca_lda_with_gnu_octave}{given on the wiki page}.

\begin{lstlisting}[language=c++]
// ...
// include the fisherfaces header
#include "fisherfaces.hpp"
// ... or 
//#include "subspace.hpp"

int main(int argc, char *argv[]) {
  // example taken from: http://www.bytefish.de/wiki/pca_lda_with_gnu_octave
  double d[11][2] = {
      {2, 3},
      {3, 4},
      {4, 5},
      {5, 6},
      {5, 7},
      {2, 1},
      {3, 2},
      {4, 2},
      {4, 3},
      {6, 4},
      {7, 6}};
  int c[11] = {0,0,0,0,0,1,1,1,1,1,1};
  // convert into OpenCV representation
  Mat _data = Mat(11, 2, CV_64FC1, d).clone();
  vector<int> _classes(c, c + sizeof(c) / sizeof(int));
  // perform the lda
  subspace::LinearDiscriminantAnalysis lda(_data, _classes);
  // GNU Octave finds the following Eigenvalue:
  //octave> d
  //d =
  //   1.5195e+00
  //
  // Eigen finds the following Eigenvalue:
  // [1.519536390756363]
  //
  // Since there's only 1 discriminant, this is correct.
  cout << "Eigenvalues:" << endl << lda.eigenvalues() << endl;
  // GNU Octave finds the following Eigenvectors:
  //  octave:13> V(:,1)
  //  V =
  //
  //     0.71169  -0.96623
  //    -0.70249  -0.25766
  //
  // Eigen finds the following Eigenvector:
  // [0.7116932742510111;
  //  -0.702490343980524 ]
  //
  cout << "Eigenvectors:" << endl << lda.eigenvectors() << endl;
  // project a data sample onto the subspace identified by LDA
  Mat x = _data.row(0);
  cout << "Projection of " << x << ": " << endl;
  cout << lda.project(x) << endl;
}
\end{lstlisting}
