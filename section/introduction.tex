\section{Introduction}

\href{http://opencv.willowgarage.com}{OpenCV (Open Source Computer Vision)} is a popular computer vision library started by \href{http://www.intel.com}{Intel} in 1999. The cross-platform library sets its focus on real-time image processing and includes patent-free implementations of the latest computer vision algorithms. In 2008 \href{http://www.willowgarage.com}{Willow Garage} took over support and OpenCV 2.3.1 now comes with a programming interface to C, C++, \href{http://www.python.org}{Python} and \href{http://www.android.com}{Android}. OpenCV is released under a BSD license, so it is used in academic and commercial projects such as \href{http://www.google.com/streetview}{Google Streetview}.

This document is the guide I've wished for, when I was working myself into face recognition. It helps you with installing OpenCV2 on your machine and explains you how to build a project on Windows and Linux. Two face recognition algorithms are prototyped with \ifx\python\undefined \href{http://www.gnu.org/software/octave/}{GNU Octave}/\href{http://www.mathworks.com}{MATLAB} \else \href{http://www.python.org}{Python}\fi{} and implemented with the OpenCV2 C++ API. All concepts are explained in detail, but a basic knowledge of C++ is assumed. I've decided to leave the C++ implementation details out (as I am afraid they confuse people) and provide you with examples how to use the projects. \href{http://www.mingw.org}{MinGW} (the GCC port for Windows) is used as the C/C++ compiler for Windows, because it works great with OpenCV2 and comes under terms of a public license (please see \href{http://www.mingw.org/license}{mingw.org/license} for details). If someone writes a similar guide for Microsoft Visual Studio 2008/2010, I would be happy to add it to the document.

You don't need to copy and paste the code snippets, the code has been pushed into my github repository:

\begin{itemize}
  \item \href{http://www.github.com/bytefish}{github.com/bytefish}
  \item \href{http://www.github.com/bytefish/facerecognition_guide}{github.com/bytefish/facerecognition\_guide}
  \item \href{http://www.github.com/bytefish/opencv}{github.com/bytefish/libfacerec}
\end{itemize}

All code is released under a \href{http://www.opensource.org/licenses/bsd-license}{BSD license}, so feel free to use it for your projects. Note: You are currently reading the \ifx\python\undefined GNU Octave/MATLAB version of this document, you can compile the Python version with \lstinline|make python|.\else Python version of this document, you can compile the GNU Octave/MATLAB version wth \lstinline|make octave|.\fi
